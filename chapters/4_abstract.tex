\cleardoublepage
\thispagestyle{empty}

\begin{center}
    \Large
    \textbf{\thetitle}
    
    \vspace{0.4cm}
    \large
    \textbf{\theauthor}
    
    \vspace{0.4cm}
    \textbf{Abstract}
\end{center}

The inflammatory myopathies are a heterogeneous family of rare autoimmune diseases affecting multiple organs and systems, including the muscle, skin, lung, and/or the joints.  Accurately defining its pathogenesis and to classify them correctly are key for understanding and managing these diseases.

This doctoral thesis has two main objectives. First, to define which are the most important pathogenic pathways and the specific expression profiles in the muscle tissue of patients with different types of myositis. Second, to develop a research framework to determine if the autoantibodies are superior to the clinical classification systems to predict the phenotype of patients with myositis.

To achieve the first objective we performed RNA sequencing on 119 muscle biopsies of patients with different types of myositis and 20 controls. We studied the differential expression, performed pathway analysis and developed exploratory machine learning pipelines to define the specific expression profiles and pathogenic pathways in each disease subgroup. To complete the second objective we developed a clinical research framework in a large myositis cohort. This framework included designing a database to store and validate the data that was entered, and developing tools to parse information from clinical reports, merge tables for analysis, and automate table and graph creation. With this framework we explored specific autoantibody-defined myositis subsets and quantitatively compared the ability of autoantibodies to the 2017 EULAR/ACR classification standard to predict the phenotype of patients with myositis.

With these studies we determined that the autoantibodies outperform current clinical criteria to predict the phenotype of myositis patients and discovered unique expression profiles in the muscle tissue of patients with different types of myositis.
