\chapter{Introduction}

\section{A brief history of myositis}

The inflammatory myopathies are a group of rare systemic autoimmune diseases characterized by variable involvement of the muscle, skin, lungs and/or joints.\cite{SelvaOCallaghan2018}

In the 19th century, Wagner described the first case of myositis.\cite{Wagner1863} Over the next years, these diseases were collectively named as \gls{pm}, myositis universalis acuta or pseudotrichinosis.\cite{Pottain1875,Hepp1887,Unverricht1887,Stertz1916,Kankeleit1916,Patjes1926,Hautveränderungen1930} At the end of the 19th century, Unverricht used the term \gls{dm} to define those patients showing both muscle and skin involvement.\cite{Unverricht1891}

During the 20th century, several key clinical observations were made in patients with myositis. Thus, it was identified that neoplasms were common in patients with \gls{dm}, that some patients had subcutaneous calcinosis and others characteristic erythematous lesions on the knuckles that were called Gottron’s papules and are pathognomonic of \gls{dm}.\cite{Stertz1916,Kankeleit1916,Patjes1926,Hautveränderungen1930} Also, in 1940 it was found that \gls{dm} was also present in children and that in juvenile cases it had an important vascular component and could lead to fatal gastrointestinal complications.\cite{Hecht1940} Finally, in 1956 it was discovered that myositis was also associated with interstitial lung disease.\cite{Mills1956}

Years later, in 1967, it was observed that some patients with \gls{pm} showed inclusion in the muscle that looked like those caused by myxovirus infections.\cite{Chou1967} This type of myositis was named \gls{ibm} and the first case series, from 1978, clearly defined that these patients were refractory to immunosuppressant medication and showed important distal weakness and marked anterior thigh compartment muscle atrophy.\cite{Carpenter1978}

In 1975 Krain\cite{Krain1975} described some patients developing the characteristic skin features of \gls{dm} without muscle involvement at the onset of the disease. Four years later, Pearson coined this phenomenon amyopathic dermatomyositis.\cite{Pearson1979} Given that Krain’s patients eventually developed muscle involvement after an amyopathic onset, it was believed that this type of patient will invariably develop muscle weakness during the first years of their evolution. However, Sontheimer and cols. proved that a subset of dermatomyositis patients never developed clinically relevant muscle involvement.\cite{Euwer1991} Later on, the term amyopathic dermatomyositis was expanded to clinically amyopathic dermatomyositis to acknowledge the fact that some of these patients did not have clinically relevant muscle weakness but had minor muscle involvement detectable by elevation of muscle enzymes, \gls{emg}, muscle biopsy or muscle \gls{mri}.\cite{Sontheimer2002}

Between 1976 and 1985, a series of studies from Reichlin, Targoff, Nishikai, Hochberg, Arnett, and others established that many patients with myositis were positive for specific autoantibodies.\cite{Nishikai1980,Targoff1985,Reichlin1976,Hochberg1984} Years later, in 1991, Dr. Lori Love and Fred Miller suggested that these autoantibodies defined more homogeneous subsets of patients and could have classificatory value.\cite{Love1991}

Finally, the last major clinical group to be added to the field of myositis was the immune-mediated necrotizing myositis. The original description in 1991\cite{EmslieSmith1991} included three cases of necrotizing myositis without significant inflammatory cell infiltration and microangiopathy with thickened “pipestem” capillaries, microvascular deposits of complement, and capillary depletion. Over the years, the vascular component of the syndrome has been deemphasized and this entity has been re-defined as presenting exclusive muscle involvement, often severe, with necrosis but without significant inflammatory infiltrates.

\section{Classification}

Given the heterogeneity of clinical manifestations and the multiplicity of serological groups in myositis, the classification of these patients has not been an easy task and it is still currently a work in progress.

Originally, Bohan and Peter suggested a single set of criteria to diagnose all patients with myositis based on the presence of muscle weakness, elevation of muscle enzymes, an irritable pattern in the \gls{emg}, an inflammatory muscle biopsy and, in those with dermatomyositis, the presence of the characteristic skin rash.\cite{Bohan1975} These criteria were highly influential and have been the gold standard to develop new criteria ever since. The main criticism to this criteria was that they did not include \gls{ibm} as a distinct category and, thus, all the \gls{ibm} patients would be classified as \gls{pm}.

Since 1975 there have been multiple systems proposed to classify patients with myositis. From those, perhaps the most influential were (Table \ref{tab:classifications}):

\begin{enumerate}
	\item Tanimoto's criteria in 1995\cite{Tanimoto1995} expanding the classificatory clinical features of Bohan and Peter's criteria.
	
	\item Griggs' criteria for \gls{ibm} in 1995\cite{Griggs1995} that required a combination of clinical, epidemiological and pathological features to establish the diagnosis of \gls{ibm}. Unfortunately, those compliant with the epidemiological and clinical criteria but with incomplete biopsy findings were labeled as "possible" \gls{ibm}. Studies tended to exclude \gls{ibm} patients falling in Griggs' "possible" criteria even if they were very similar to "probable" \gls{ibm} categorizations using subsequent classification proposals.
	
	\item Targoff's ingenious criteria in 1997\cite{Targoff1997} including the \gls{msa} in a manner that would allow the criteria to be automatically updated as new autoantibodies were discovered.
	
	\item Badrising's criteria in 2000\cite{Badrising2000} modifying a previous set of criteria from the \gls{enmc} in 1997.\cite{Emery1997} Requires muscle weakness and a mononuclear inflammatory infiltrates with invasion of non-necrotic muscle fibers combined with either a set of clinical features or a combination of clinical and pathologic features. Includes two categories: probable and definite.
	
	\item Sontheimer's criteria in 2002\cite{Sontheimer2002} proposing to consider amyopathic \gls{dm} as a new major clinical group.
	
	\item Dalakas and Hohlfeld criteria in 2003\cite{Dalakas2003} which were heavily reliant on the muscle biopsy findings to do the diagnosis of most myositis subtypes.
	
	\item 119th \gls{enmc} criteria in 2004,\cite{Hoogendijk2004} also heavily reliant on the muscle biopsy findings.
	
	\item First Medical Research Council workshop on \gls{ibm} in 2010\cite{HiltonJones2010} proposed a set of \gls{ibm} criteria where pathologically defined \gls{ibm} equal to Griggs'. Both clinically defined and possible \gls{ibm} required the presence of either rimmed vacuoles, increased \gls{mhc}-I, or invasion of non-necrotic fibers by mononuclear cells plus a set of clinical features.
	
	\item Pestronk's criteria in 2011,\cite{Pestronk2011} proposing a novel classification exclusively based on histologic findings.
	
	\item 188th \gls{enmc} \gls{ibm} criteria in 2011\cite{Rose2013} requiring a set of clinical and epidemiological features accompanied by the characteristic pathological features.
\end{enumerate}

Importantly, in 2017 were published the joint EULAR/ACR criteria for myositis.\cite{Lundberg2017} These criteria used a weighted score based on a set of epidemiologic, clinical, and laboratory variables to classify patients as myositis. Those patients classified as myositis could be further subclassified in four different categories: \gls{pm}/\gls{imnm}, \gls{ibm}, \gls{adm}, \gls{dm}, juvenile dermatomyositis, and juvenile myositis other than JDM.\cite{Lundberg2017}

Notwithstanding the support of the two main rheumatologic associations in the world, the 2017 EULAR/ACR myositis criteria had multiple methodological problems that limited its applicability in clinical research:

\begin{enumerate}
	\item Although it was accepted in the field that \gls{msa} help to define phenotypically distinct sets of myositis patients, the 2017 EULAR/ACR myositis classification criteria included only anti-Jo1 autoantibodies (of note, the presence of anti-Jo1 antibodies was the variable with the highest value to classify a patient as myositis).
	
	\item These criteria did not classify patients better than Targoff's criteria and based on the composition of their artificial cohort of patients it is not clear that they would outperform other proposals in real clinical settings (e.g. for a cohort with of patients with a prevalence of myositis of 90\% Bohan and Peter would classify correctly more patients than the 2017 EULAR/ACR classification).
	
	\item The criteria were developed using a data-driven approach but had to be modified based on the opinion of experts because they did not fit the current state of the art.
	
	\item It was used as a convenience sampling to develop the criteria, which makes the interpretation of the probability scores complicated and invalidates the predictive values reported in the manuscript.
	
	\item The gold-standard to build the classification was the opinion of experts and, thus, the data-driven approach was in reality equivalent to the opinion of experts that they used as a starting point.
	
	\item The complexity of the criteria made it hard to use and even more complicated to memorize.
	
	\item Some of the variables lacked proper definition. For example, the lack of response to treatment in the classificatory tree.
	
	\item Finally, the cutoff was selected by experts and there was no complete external validation of the criteria (only a validation of the sensitivity).
\end{enumerate}

After the 2017 EULAR/ACR criteria were published, the \gls{enmc} sponsored a series of workshops to develop criteria for each one of the main groups of patients with myositis. Thus, in 2017 the \gls{enmc} published their proposal to classify \gls{imnm}\cite{Allenbach2017} and in 2019 another to classify \gls{dm}.\cite{Mammen2020} Importantly, these two sets of criteria heavily weighted on the importance of \gls{msa}, and for the first time, allowed patients positive for \gls{msa} to be diagnosed as \gls{imnm} or \gls{dm} requiring only that they showed either muscle or skin involvement, respectively. Unfortunately, by being developed independently, both the \gls{imnm} and the \gls{dm} criteria were not mutually exclusive and it would be feasible for a patient to fulfill both of them simultaneously, which complicates classificatory tasks.

Also, in 2018, a french group proposed a new set of criteria based on performing unsupervised multiple correspondence analysis and hierarchical clustering to aggregate patients in subgroups.\cite{Mariampillai2018} However, most of the study is focused on describing the characteristics of the artificial clusters that their methodology defined. Besides methodologic concerns on the methodology that they use to determine the number of clusters,\cite{PinalFernandez2019} the classification criteria that they propose are unrealistically simplistic, using only the presence of \gls{dm} rash, presence of antisynthetase autoantibodies and finger flexor weakness to classify patients. Moreover, these criteria are questionable from a practical standpoint since, for example, they would classify a patient with anti-Jo1 autoantibodies and \gls{dm} rash as the cluster corresponding to dermatomyositis and not as an antisynthetase syndrome.

\begin{table}
	\caption{Most influential classification criteria in myositis.}
	\resizebox{\textwidth}{!}{
	\begin{tabular}{|l|l|p{.75\textwidth}|}
		\hline
		Authors                               & Year & Characteristics                                                                                                                                                         \\ \hline\hline
		Bohan and Peter\cite{Bohan1975}       & 1975 & Most influential classification in myositis, based on the presence of a combination of clinical and laboratory findings.                                                \\
		Tanimoto\cite{Tanimoto1995}           & 1995 & Expanded the classificatory clinical features of Bohan and Peter criteria.                                                                                              \\
		Griggs\cite{Griggs1995}               & 1995 & Most influential criteria for \glsfirst{ibm}. Based on a combination of clinical, epidemiological and pathological features.                                            \\
		Targoff\cite{Targoff1997}             & 1997 & Early criteria including the \glsfirst{msa}.                                                                                                                            \\
		Badrising\cite{Badrising2000}         & 2000 & Modified version of a previous set of criteria from the \gls{enmc} in 1997.                                                                                             \\
		Sontheimer\cite{Sontheimer2002}       & 2002 & Criteria proposing to include amyopathic \glsfirst{dm} as a new major clinical group.                                                                                   \\
		Dalakas and Hohfeld\cite{Dalakas2003} & 2003 & Heavily reliant on the muscle biopsy findings to do the diagnosis of most myositis subtypes.                                                                            \\
		Hoogendijk\cite{Hoogendijk2004}       & 2004 & Also heavily reliant on the muscle biopsy findings.                                                                                                                     \\
		Hilton-Jones\cite{HiltonJones2010}    & 2010 & \gls{ibm} criteria. Pathologically defined \gls{ibm} equal to Griggs', both clinically defined and possible \gls{ibm} require the characteristic pathological features. \\
		Pestronk\cite{Pestronk2011}           & 2011 & Classification exclusively based on muscle biopsy findings.                                                                                                             \\
		Rose\cite{Rose2013}                   & 2013 & \gls{ibm} criteria requiring a set of clinical and epidemiological features accompanied by the characteristic pathological features.                                    \\
		Lundberg\cite{Lundberg2017}           & 2017 & Current EULAR/ACR consensus criteria. Considerable methodological issues.                                                                                               \\
		Allenbach\cite{Allenbach2017}         & 2017 & \Glsdesc{imnm} criteria emphasizing the importance of \gls{msa}.                                                                                                        \\
		Mariampillai\cite{Mariampillai2018}   & 2018 & Concerns about the methodology. Unrealistically simplistic and questionably practical.                                                                                  \\
		Mammen\cite{Mammen2020}               & 2020 & \gls{dm} criteria emphasizing the importance of \gls{msa}.                                                                                                              \\ \hline
	\end{tabular}}
	\label{tab:classifications}
\end{table}

Persistent unsolved controversies among experts regarding myositis classification include:
 
\begin{enumerate}
	\item If the different criteria should be used exclusively for research studies or if they should aim to be also useful for clinical diagnosis.
	
	\item If it should be only one set of criteria to fit all types of myositis or one for each type of myositis.
	
	\item If only one set of criteria is used, if the criteria should only try to distinguish myositis vs. non-myositis patients or they should also define the myositis subgroup.
	
	\item If criteria are developed individually for each type of myositis how can we ensure that they are mutually exclusive. 
\end{enumerate}

\section{Major myositis subgroups, pathogenesis, and autoantibodies}

In this section I will review the features of the main clinical subgroups currently recognized within myositis: \gls{ibm}, \gls{imnm}, \gls{dm}, overlap myositis (including the \gls{as}), and \gls{pm} (Table \ref{tab:autoantibodies}). Most \gls{msa} are generally associated with one of these broad clinical subgroups (except for \gls{ibm} that does not have any known \gls{msa}). Thus, within each section, I will review the most relevant autoantibodies that have been described so far. Moreover, in the last section, I will briefly discuss a group of autoantibodies that are not specific for a particular clinical phenotype but may act as disease modifiers.

\begin{table}
	\caption{Clinical features and grouping of the most main \glsdesc{msa}.}
	\resizebox{\textwidth}{!}{
	\begin{tabular}{|l|c|c|c|}
		\hline
		Group                 & Muscle &     Lung      &     Skin      \\ \hline\hline
		\Glsdesc{ibm}         &  +++   & $\varnothing$ & $\varnothing$ \\
		\Glsdesc{imnm}        &        &               &               \\
		\quad Anti-SRP        &  +++   &       +       & $\varnothing$ \\
		\quad Anti-HMGCR      &  +++   & $\varnothing$ & $\varnothing$ \\
		\Glsdesc{dm}          &        &               &               \\
		\quad Anti-Mi2        &   ++   & $\varnothing$ &      ++       \\
		\quad Anti-NXP2       &   ++   & $\varnothing$ &      ++       \\
		\quad Anti-TIF1       &   +    & $\varnothing$ &      ++       \\
		\quad Anti-SAE        &   +    & $\varnothing$ &      ++       \\
		\quad Anti-MDA5       &   +    &      +++      &      +++      \\
		Overlap myositis      &        &               &               \\
		\quad \Glsdesc{as}    &        &               &               \\
		\quad \quad	Anti-Jo1  &   ++   &      ++       &       +       \\
		\quad \quad	Anti-PL7  &   ++   &      +++      &       +       \\
		\quad \quad	Anti-PL12 &   +    &      +++      &       +       \\
		\quad Anti-Pm/Scl     &   +    &       +       &       +       \\
		\quad Anti-Ku         &   +    &       +       &       +       \\
		\quad Anti-U1RNP      &   +    &       +       &       +       \\ \hline
	\end{tabular}}
	\label{tab:autoantibodies}
\end{table}

\subsection{Sporadic inclusion body myositis}

As with other myositis types, \gls{ibm} patients show muscle weakness and are usually found to have elevated \gls{ck} levels and myopathic \gls{emg} features. However, \gls{ibm} patients are usually over 50 years-old while the other myositis can also develop in younger patients, including children.\cite{SelvaOCallaghan2018} Also, other myositis are more frequent in women, but in \gls{ibm} men are affected twice as frequently as women.\cite{SelvaOCallaghan2018}.

Regarding the pattern of muscle weakness, patients with \gls{ibm} usually have distal weakness, including the finger flexors, wrist flexors, and ankle dorsiflexors, which is rarely prominent for other types of myositis.\cite{Lloyd2014,SelvaOCallaghan2018} Also, symmetric weakness is the rule in patients with other types of myositis, but many \gls{ibm} patients have an asymmetric pattern of weakness.\cite{SelvaOCallaghan2018} Moreover, weakness can occur over weeks or months in other myositis, but the course of the disease in \gls{ibm} is usually slow with weakness occurring over the course of years.\cite{SelvaOCallaghan2018} Finally, compared with other myositis, \gls{ibm} patients have the most characteristic muscle \gls{mri} pattern, with severe involvement of the anterior thigh compartment.\cite{Tasca2015,PinalFernandez2017}

Importantly, there is no clear evidence that immunosuppression benefits patients with \gls{ibm} whereas other myositis usually do respond to immunosuppression.\cite{SelvaOCallaghan2018} Moreover, \gls{ibm} patients also present progressive dysphagia,\cite{SelvaOCallaghan2018} that can lead to bronchoaspiration and can be studied using videofluoroscopy. 

Also, unlike other types of myositis, \gls{ibm} is not associated with any \gls{msa}. Notwithstanding this, autoantibodies recognizing NT5C1a are present in 30-60\% of \gls{ibm} patients, but they are also found in 5-10\% PM and 15-20\% of \gls{dm} patients and patients with lupus and Sjögren's syndrome.\cite{Lloyd2016,Herbert2016,Muro2017,Lilleker2017} Anti-NT5C1a autoantibodies have been associated with increased severity and mortality in these patients.\cite{Lilleker2017,Goyal2016} Additionally, a recent report has suggested that anti-NT5C1a autoantibodies may directly cause muscle damage.\cite{Tawara2017}

As for the muscle biopsies of patients with \gls{ibm}, they characteristically include co-existing inflammation, abnormal protein aggregation, and mitochondrial dysfunction.\cite{Dalakas2002} The inflammatory infiltrate is comprised of CD8+ T cells that surround and invade non-necrotic fibers (a.k.a primary inflammation). Importantly, it has been recently found that these terminally differentiated T-cells express the surface marker KLRG1 and the carbohydrate epitope CD57, and that in most (22/38 [58\%]) patients these cells meet criteria for T-cell large granular lymphocytic leukemia.\cite{Greenberg2016,Greenberg2019} Although this association is yet to be confirmed, it would explain the refractoriness and advanced age of these patients.\cite{Greenberg2016,Greenberg2019,Greenberg2019a}

Rimmed vacuoles, best visualized by Gomori trichrome staining, are a hallmark of \gls{ibm} muscle biopsies. Although some patients with hereditary myopathies also have rimmed vacuoles, their presence can help in distinguishing \gls{ibm} from other myositis.\cite{Dalakas2002} How \gls{ibm} rimmed vacuoles are formed remains unknown. However, nuclear membrane proteins are found within rimmed vacuoles, suggesting they could be the remnants of degenerated myonuclei.\cite{Greenberg2006,Nalbantoglu1994} A more recent study revealed that proteins accumulating in rimmed vacuoles are related to protein folding and autophagy, suggesting that impaired autophagic function may be implicated in their formation.\cite{Guttsches2017} Cytoplasmic accumulations also contain Congo red staining material ("amyloid"), p62 and TDP-43.\cite{Dalakas2002} Although it is a widely spread notion that the cytoplasmic inclusions contain $\beta$-amyloid, studies specifically measuring the expression of this protein in the muscle of patients with \gls{ibm} failed to prove this fact.\cite{Greenberg2010,Nalbantoglu1994}

An increased number of cytochrome oxidase negative muscle fibers and the presence of “ragged red fibers” suggest that mitochondrial damage plays a significant role in \gls{ibm}.\cite{Dalakas2002} Accordingly, a recent study showed that mitochondrial DNA is depleted and that mitochondrial fusion proteins are dysregulated in \gls{ibm} muscle.\cite{CatalanGarcia2016} Furthermore, an increased frequency of mitochondrial DNA deletions has been reported in \gls{ibm} muscle.\cite{Rygiel2016}

Important in \gls{ibm}, but relevant to all types of myositis, performing a muscle \gls{mri} to select the location of the muscle biopsy increases the diagnostic accuracy of the pathology.\cite{VanDeVlekkert2015}

\subsection{Immune-mediated necrotizing myositis}

\gls{imnm} can be described as a distinct type of myositis characterized by proximal muscle weakness, exceptionally high muscle enzyme levels, myopathic \gls{emg} findings, and muscle biopsies showing necrosis and/or regeneration with minimal lymphocytic infiltrates and no perifascicular atrophy. Typical \gls{imnm} muscle biopsies also include \gls{mhc} type I upregulation, M2-macrophage infiltration and \gls{mac} deposition on non-necrotic fibers.\cite{Watanabe2016,Chung2015} Extramuscular manifestations are rare and generally mild when they occur.\cite{PinalFernandez2017b,Suzuki2015,Tiniakou2017}

Around two-thirds of the patients with \gls{imnm} have autoantibodies recognizing either the signal recognition particle (SRP) or HMG-CoA reductase (HMGCR). However, ~20\% anti-SRP-positive and anti-HMGCR-positive patients have lymphocytic infiltrates on their muscle biopsies but are otherwise indistinguishable from their counterparts with necrotizing biopsies.\cite{Suzuki2015,Mammen2011,Allenbach2018}

Anti-SRP and anti-HMGCR myopathy share many features, including similar muscle biopsy findings and high \gls{ck} levels, and minimal extramuscular manifestations.\cite{Watanabe2016} Furthermore, in both, younger patients seem to have more aggressive and refractory muscle disease.\cite{PinalFernandez2017b,Tiniakou2017} However, differences between these two \gls{imnm} subtypes have been documented. First, anti-HMGCR myopathy is associated with statin exposure,\cite{ChristopherStine2010} while anti-SRP myopathy is not associated with statins.\cite{PinalFernandez2017b,Suzuki2015} Second, anti-SRP-positive patients have more severe weakness and a higher number of necrotic muscle fibers than anti-HMGCR-positive patients.\cite{PinalFernandez2017b,Watanabe2016,Allenbach2018} Third, the presence of interstitial lung disease, although uncommon in both groups, is more common in those with anti-SRP autoantibodies (13-22\%) than in anti-HMGCR (<5\%).\cite{PinalFernandez2017b,Suzuki2015,Watanabe2016,Tiniakou2017} Fourth, a single report suggested that anti-HMGCR myopathy and autoantibody-negative \gls{imnm} may have an increased risk of malignancy.\cite{Allenbach2016} However, autoantibody-negative \gls{imnm} is an ill-defined entity and other cohorts of anti-HMGCR patients did not confirm an association with cancer in this type of patients.\cite{Tiniakou2017,Watanabe2016} Fifth, several studies have confirmed DRB1*11:01 as an immunogenetic risk factor for developing anti-HMGCR myopathy (present in ~70\% of those with anti-HMGCR autoantibodies but only in ~15\% of the general population) and one report suggested that class II \gls{hla} allele DRB1*08:03 is associated with anti-SRP myopathy.\cite{Ohnuki2016,Mammen2012,Limaye2015} Finally, anti-HMGCR myopathy has rarely been associated with cardiac involvement.\cite{Watanabe2016} In contrast, early cross-sectional studies in anti-SRP suggested a high prevalence of cardiac manifestations in these patients,\cite{Targoff1990,Kao2004} although this has not been confirmed in recent cohort studies.\cite{PinalFernandez2017b,Suzuki2015,Watanabe2016} In patients with suspicion of cardiac involvement an electrocardiogram and an echocardiogram should be performed. A gadolinium-enhanced MRI can assess for active myocardial inflammation and in selected cases, an endomyocardial biopsy can confirm the diagnosis.\cite{Chen2018}

The mechanisms underlying myofiber necrosis in \gls{imnm} remains to be elucidated. However, some clues have emerged. For example, given the \gls{mac} deposits on the surface of non-necrotic fibers, it has been proposed that anti-SRP and anti-HMGCR autoantibodies could be directly pathogenic.\cite{Allenbach2018} In this regard, a recent study suggested that these autoantibodies may induce muscle atrophy, increase levels of reactive oxygen species and cytokines (e.g., tumor necrosis factor and IL-6), and impair myoblast fusion (by decreasing the production of IL-4 and IL-13) of cultured muscle cells.\cite{AroucheDelaperche2017} However, these \gls{imnm}-associated autoantibodies did not induce necrosis and further studies may be needed to show that they are pathogenic in vivo.\cite{AroucheDelaperche2017}

\subsection{Dermatomyositis}

\gls{dm} patients classically show proximal muscle weakness and characteristic cutaneous manifestations that develop over weeks to months. However, some patients with \gls{dm} rash have little or no muscle involvement as demonstrated by lack of weakness, muscle enzyme elevation, \gls{mri}, \gls{emg}, or muscle biopsy findings. Clinically amyopathic if often considered as a different subtype of myositis\cite{Sontheimer2002} but, for simplicity, we will include them in this section.

The pathognomonic skin rash of \gls{dm} includes violaceous periorbital, often edematous, rash (i.e., heliotrope rash) as well as erythematous lesions on the extensor surfaces of the joints (i.e., Gottron’s papules). Usually, muscle enzymes are elevated and the \gls{emg} reveals a myopathic pattern (myopathic motor units with fibrillations and spontaneous sharp waves). As in other types of myositis, the \gls{mri} in \gls{dm} may reveal intramuscular T2 hyperintensities caused by muscle inflammation and/or necrosis.\cite{PinalFernandez2017} In addition, \gls{dm} patients often have T2 hyperintensities around individual muscles as a result of fascial involvement, a feature seen less frequently in other myositis.\cite{PinalFernandez2017} 

Perifascicular atrophy is a highly specific feature of muscle biopsies from \gls{dm} patients (specificity >90\%),21 but it lacks sensitivity (25-50\%).\cite{SuarezCalvet2017,PinalFernandez2015} Limited data support that perifascicular MX1 (human myxovirus resistance protein 1) and RIG-1 (retinoic acid-inducible gene I) expression have higher diagnostic sensitivity (71\% and 50\%) than perifascicular atrophy in \gls{dm}.\cite{SuarezCalvet2017,Uruha2017} Additionally, \gls{dm} biopsies often have cellular infiltrates consisting predominantly of CD4+ T cells, plasmacytoid dendritic cells, B cells, and macrophages.\cite{Dalakas2002} These cells often surround medium-sized blood vessels (perivascular inflammation) and invade the perimysium.\cite{Dalakas2002} However, up to 16\% of \gls{dm} biopsies lack infiltrates and have prominent necrosis that is pathologically indistinguishable from \gls{imnm}.\cite{PinalFernandez2015} Capillary loss can occur and it can be detected deposition of \gls{mac} and presence of microtubular inclusions on intramuscular capillaries.\cite{Dalakas2002} Furthermore, as in other myositis, there is usually upregulation of class I \gls{mhc} on the sarcolemma of muscle fibers. In \gls{dm} patients, class I \gls{mhc} upregulation, and other pathological findings (e.g., myofiber de/regeneration and necrosis) may be especially prominent in perifascicular regions.\cite{Dalakas2002}

Approximately 70\% of \gls{dm} patients have one \gls{msa}.\cite{Betteridge2016} Each \gls{dm}-specific autoantibody is associated with a unique clinical phenotype. Thus, autoantibodies recognizing Mi2 have been associated with “classic” \gls{dm} features including proximal muscle weakness and severe skin manifestations.\cite{Ghirardello2005} \gls{dm} patients with autoantibodies recognizing nuclear matrix protein (NXP)-2 are more likely than other \gls{dm} patients to present with both proximal and distal muscle weakness, subcutaneous edema, and/or dysphagia.\cite{Albayda2017} Furthermore, anti-NXP2-positive patients are more prone than other \gls{dm} patients to develop calcinosis, which are painful deposits of calcium in the soft tissues, often refractory to immunosuppressant treatment.\cite{Albayda2017} \gls{dm} patients with anti-transcription intermediary factor (TIF)-1$\gamma$ and, to a lesser degree, those with anti-NXP2 autoantibodies are at increased risk of malignancy within three years of their diagnosis; as such, these patients may require comprehensive cancer screening.\cite{Albayda2017,Fiorentino2013,TralleroAraguas2012} The traditional approach to cancer screening is to perform a complete physical examination, general laboratory tests, tumor markers, thoracoabdominal \gls{ct}, and a gynecologic exam, including ultrasonography and mammography plus any other age and gender-appropriate screening tests. Alternatively, a single \gls{pet}-\gls{ct} has been shown to have an equivalent sensitivity for detecting malignancy as the traditional approach.\cite{SelvaOCallaghan2010}

Patients with \gls{dm} and autoantibodies recognizing small ubiquitin-like modifier activating enzyme (SAE) or melanoma differentiation-associated gene 5 (MDA5) tend to have more significant skin than muscle involvement.\cite{Ge2017,LabradorHorrillo2014,Sato2009,Narang2015} Along with the typical skin manifestations of \gls{dm}, anti-MDA5-positive patients are prone to develop ulcers, often on the flexor surface of the digits and palm.\cite{Narang2015} Most anti-MDA5 patients are hypo or amyopathic.\cite{LabradorHorrillo2014,Sato2009,Narang2015} Furthermore, unlike patients with other \gls{dm} autoantibodies, anti-MDA5-positive patients frequently develop a rapidly progressive and sometimes lethal form of interstitial lung disease (\gls{ild}).\cite{LabradorHorrillo2014,Sato2009} All myositis patients with suspicion of \gls{ild} should initially be evaluated using pulmonary function tests (including CO diffusion and ins/expiratory pressures) and a \gls{hrct}. \gls{ild} monitoring should rely on periodical PFTs and subsequent \gls{hrct} should be restricted to evaluating those with evolving pulmonary issues.

Some combination of genetic risk factors and environmental exposures are presumably required to trigger \gls{dm}. Indeed, several immunogenetic risk factors, including certain class II \gls{hla} alleles, have been implicated in \gls{dm} pathogenesis.\cite{Miller2015} Interestingly, ultraviolet light exposure is also a known risk factor for developing \gls{dm}.\cite{Mamyrova2017} However, the majority of people with known genetic risk factors, even those with high ultraviolet light exposure, never develop \gls{dm}. An increased number of mutations and loss of heterozygosity in TIF1 genes from tumors in anti-TIF1$\gamma$-positive \gls{dm} patients have recently been reported.\cite{PinalFernandez2018} This observation suggests the possibility that mutations in TIF1 genes may generate neoantigens that could trigger autoimmunity by means of molecular mimicry.

Whatever the cause, once a patient has developed \gls{dm}, it’s unclear what mechanisms maintain muscle damage and weakness. Notwithstanding this, there is strong evidence that the \gls{ifn} pathway is relevant to DM pathogenesis.\cite{Greenberg2005} Specifically, a marked overexpression of \gls{ifn}-inducible genes has been demonstrated in the muscle,\cite{Greenberg2005} peripheral blood,\cite{Walsh2007,Baechler2007} and skin\cite{Wong2012} of DM patients. Moreover, the expression levels of \gls{ifn}-inducible genes correlate with indicators of DM disease activity.\cite{Walsh2007,Baechler2007} The presence of plasmacytoid dendritic cells, potent sources of interferon, along with the increased expression of type-I-interferon-inducible proteins in the perifascicular area, suggest that interferon could somehow mediate perifascicular atrophy.\cite{SuarezCalvet2017,Greenberg2005} 

\subsection{Overlap myositis}

Autoimmune myopathy may also occur in patients presenting features of other autoimmune disease, such as lupus, rheumatoid arthritis, Sjögren syndrome or systemic sclerosis.\cite{GuillenDelCastillo2014,Rigolet2012,EscolaVerge2017} Many of these patients also have autoantibodies that are associated with characteristic phenotypes.\cite{GuillenDelCastillo2014,Rigolet2012,EscolaVerge2017}

The most representative form of overlap myositis is antisynthetase syndrome (\gls{as}), with autoantibodies targeting the aminoacyl tRNA synthetases, enzymes that conjugate an amino acid to its cognate tRNA.\cite{PinalFernandez2017a,TralleroAraguas2016} Those recognizing histidyl-tRNA synthetase (anti-Jo1), threonyl-tRNA synthetase (anti-PL7), and alanyl-tRNA synthetase (anti-PL12) are most common.\cite{PinalFernandez2017a,TralleroAraguas2016} Patients with any one of these autoantibodies can be defined as having \gls{as} and typically present with one or more of the following features: myositis, \gls{ild}, arthritis, Raynaud syndrome, fever, and hyperkeratotic radial fingers lesions known as “mechanic’s hands”.\cite{SelvaOCallaghan2018} \gls{as} patients may also have skin rashes similar to \gls{dm}.58 Of note, not all \gls{as} patients have muscle weakness. Indeed, whereas ~90\% anti-Jo1 patients have myositis, up to 50\% anti-PL12 patients present with \gls{ild} but no muscle involvement.\cite{PinalFernandez2017a} Furthermore, anti-Jo1-positive patients have more severe weakness while anti-PL7 and anti-PL12 have more severe lung involvement.\cite{PinalFernandez2017a,TralleroAraguas2016}

When present, myopathic \gls{as} features are very similar to \gls{dm}, including proximal muscle weakness, elevated muscle enzymes, and myopathic \gls{emg}.\cite{SelvaOCallaghan2018} \gls{as} patients often have intramuscular T2 MRI hyperintensities, but a specific MRI pattern has not been described.\cite{Andersson2017} Muscle biopsies may reveal perifascicular atrophy similar to \gls{dm}. However, compared to \gls{dm}, \gls{as} may have an increased number of perifascicular necrotic fibers.\cite{MescamMancini2015,Noguchi2017} Furthermore, it has been reported that \gls{as} biopsies show nuclear actin aggregation, an electron microscopy feature that is not seen in other myositis.\cite{Stenzel2015} To date, very little is known about what triggers and maintains autoimmunity in the \gls{as}.

Anti-PM/Scl autoantibodies are associated with myositis in patients with systemic sclerosis.\cite{GuillenDelCastillo2014} Similarly, anti-U1RNP-positive mixed-connective tissue disease patients and or anti-Ku may have myositis along with additional systemic sclerosis (e.g. sclerodactyly) and lupus (e.g. glomerulonephritis or serositis) features.\cite{Rigolet2012,EscolaVerge2017}

\subsection{Polymyositis}

\gls{pm} is defined by the presence of muscle weakness, elevated \gls{ck} levels, myopathic \gls{emg} features and an inflammatory muscle biopsy with none of the characteristic accompanying features of the other abovementioned groups. Many patients previously classified as having \gls{pm} could now be considered to have \gls{as} without a rash, \gls{imnm}, or \gls{ibm} based on characteristic clinical manifestations, serological features, and muscle biopsy findings.\cite{Meulen2003,Chahin2008,Vilela2015} Even if some true \gls{pm} patients may still exist,\cite{Amato2003a} \gls{pm} remains a diagnosis of exclusion and \gls{pm} patients should be closely monitored for new clinical features suggesting alternative diagnoses.

\subsection{Myositis-associated autoantibodies}

Myositis autoantibodies have been traditionally classified in \gls{msa} and \gls{maa} depending on their association to "pure" forms of myositis (e.g. anti-Mi2, anti-NXP2, or anti-SRP) or to myositis accompanied by features of other autoimmune diseases (e.g. anti-PM/Scl, anti-Ku, or anti-U1RNP) respectively.\cite{McHugh2018} However, the term \gls{maa} has also been used for a group of autoantibodies that appear concomitantly with others in patients with very different clinical phenotypes, often acting as disease modifiers. Among this type of \gls{maa} the most common is anti-Ro52,\cite{PinalFernandez2017a,Vancsa2009,Bauhammer2016,Marie2012} but also anti-FHL1,\cite{Albrecht2015} anti-cortactin,\cite{LabradorHorrillo2014a} anti-PUF60,\cite{Fiorentino2016,Zhang2018} or the abovementioned anti-NT5c1a.\cite{Lloyd2016,Herbert2016,Muro2017,Lilleker2017,Goyal2016}

Anti-Ro52 often co-occurs with anti-Jo1 autoantibodies and patients with both antibodies have more frequent and severe \gls{ild} poorer response to immunosuppressive drugs and decreased survival.\cite{PinalFernandez2017a,Vancsa2009,Bauhammer2016,Marie2012} Moreover, high anti-Ro52 titers are associated with more severe ILD, myositis and joint involvement in anti-Jo1 patients.\cite{PinalFernandez2017a,Vancsa2009,Bauhammer2016,Marie2012} Alternatively, anti-FHL1 is present in up to 25\% of patients with myositis and is associated with the presence of muscle atrophy, dysphagia, pronounced muscle fiber damage and vasculitis.\cite{Albrecht2015} As for the anti-cortactin antibodies, they were found in 12\% of the patients with myositis, were more common in \gls{pm} or \gls{imnm} and were not associated with any specific clinical feature.\cite{LabradorHorrillo2014a} Anti-PUF60 antibodies were found in 15\% of patients with myositis, were associated with anti-TIF1g autoantibodies and were associated with a higher prevalence of skin ulcerations.\cite{Fiorentino2016,Zhang2018} Finally, as it was mentioned earlier, anti-NT5c1a is found predominantly in patients with \gls{ibm} and has been associated with increased severity and mortality in these patients.\cite{Lloyd2016,Herbert2016,Muro2017,Lilleker2017,Goyal2016}

The current terminology of \gls{msa} and \gls{maa} is rather confusing since the \gls{maa} category groups together two different populations of autoantibodies, those that are specific of a certain phenotype and those that are not. Also, establishing which phenotypes constitute "pure" forms of myositis is often complicated and of questionable practical importance. An alternative way to classify autoantibodies conceptually and solve the above-mentioned issues would be to consider them as disease-specific or disease-independent. Disease-specific antibodies would include all the \gls{msa} and those \gls{maa} that are mutually exclusive (the presence of one is associated with the absence of others) and linked to a specific phenotype (e.g. anti-PM/Scl, anti-Ku, or anti-U1RNP). Alternatively, disease-independent antibodies would be those that may act as disease modifiers but are not linked to any particular combination of clinical features (e.g. anti-Ro52, anti-FHL1, anti-cortactin, anti-PUF60, or anti-NT5c1a).
