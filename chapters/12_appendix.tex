\appendix
\chapter{Published works}
\label{chap:appendix}

\section{Muscular and extramuscular features of myositis patients with anti-U1-RNP autoantibodies. Neurology 2018.}
\label{sec:u1rnp}
In this longitudinal cohort study study we define the clinical phenotype of patients with myositis with anti-U1RNP autoantibodies. We analyzed the prevalence and severity of clinical features at disease onset and during follow-up in patients with anti-U1RNP myositis were compared to those with \gls{dm}, \gls{imnm}, and the \gls{as}.

Twenty anti-U1RNP patients, 178 patients with \gls{dm}, 135 patients with \gls{imnm}, and 132 patients with \gls{as} were included. Anti-U1RNP patients were younger (~37 years) and more likely to be black (60\%) than patients with \gls{as}, \gls{dm}, or \gls{imnm}. Muscle weakness was a presenting feature in 15\% of anti-U1RNP patients; 80\% eventually developed weakness. Four of 7 anti-U1RNP patients had necrotizing muscle biopsies. Arthritis occurred in 60\% of anti-U1RNP patients; this was increased compared to \gls{dm} (18\%) or \gls{imnm} (6\%) (all p < 0.01). \gls{dm}-specific skin features developed in 60\% of anti-U1RNP patients. \gls{ild} occurred in 45\% of anti-U1RNP patients; fewer patients with \gls{dm} (13\%) and \gls{imnm} (6\%) and more patients with \gls{as} (80\%) developed \gls{ild} (all p < 0.01). Glomerulonephritis and pericarditis occurred in 25\% and 40\% of anti-U1RNP patients, respectively, but rarely in the other groups; these features occurred only in those with coexisting anti-Ro52 autoantibodies. No anti-U1RNP patient had cancer-associated myositis or died during the study period.

In conclussion, patients with anti-U1RNP myositis typically present with proximal weakness and necrotizing muscle biopsies. Arthritis, dermatitis, and \gls{ild} are the most common extramuscular clinical features. Pericarditis and glomerulonephritis are uniquely found in patients with anti-U1RNP myositis.

\includepdf[pages=-]{pdf/u1rnp.pdf}

\section{Efficacy and adverse effects of methotrexate compared with azathioprine in the Antisynthetase Syndrome. Clin Exp Rheumatol 2019.}
\label{sec:mtx_aza}
In this study we analyzed the efficacy in terms of muscle strength, and corticosteroid tapering as well as the prevalence of adverse effects in patients with the \gls{as} treated with \gls{aza} compared to those treated with \gls{mtx}.

We compared the clinical outcomes in \gls{as} patients treated with \gls{aza} versus \gls{mtx} including change in corticosteroid dose, strength, and \gls{ck} as well as the prevalence of adverse effects.

Among 169 patients with \gls{as}, 102 were treated at some point exclusively with either \gls{aza} or \gls{mtx} (± corticosteroids). There were no significant differences in the rate of muscle strength recovery, \gls{ck} decrease or corticosteroid tapering between those \gls{as} patients treated with \gls{mtx} versus \gls{aza}. The prevalence of adverse events in patients treated with \gls{aza} and \gls{mtx} was similar (29\% vs. 25\%, p>0.05); elevated liver enzymes (17\% \gls{aza} vs. 12\% \gls{mtx}) and gastrointestinal involvement (10\% \gls{aza} vs. 8\% \gls{mtx}) were the most common adverse events. While no patients treated with \gls{aza} developed lung complications, two of the patients treated with \gls{mtx} experienced reversible pneumonitis with \gls{mtx} cessation. 

In conlussion, \gls{aza} and \gls{mtx} showed similar efficacy and adverse events in patients with \gls{as}. Pneumonitis is a rare but important event in patients receiving \gls{mtx}.

\includepdf[pages=-]{pdf/mtx_vs_aza.pdf}

\section{Anti-Ro52 autoantibodies are associated with interstitial lung disease and more severe disease in patients with juvenile myositis. Ann Rheum Dis 2019.}
\label{sec:ro52}
Anti-Ro52 autoantibodies are associated with more severe \gls{ild} in adult myositis patients with \gls{as} autoantibodies. However, few studies have examined anti-Ro52 autoantibodies in juvenile myositis. The purpose of this study was to define the prevalence and clinical features associated with anti-Ro52 autoantibodies in a large cohort of patients with juvenile myositis.

For this, we screened sera from 302 patients with juvenile \gls{dm}, 25 patients with juvenile \gls{pm} and 44 patients with
juvenile connective tissue disease–myositis overlap for anti-Ro52 autoantibodies by ELISA. Clinical characteristics were compared between myositis patients with and without anti-Ro52 autoantibodies.

Anti-Ro52 autoantibodies were found in 14\% patients with juvenile \gls{dm}, 12\% with juvenile \gls{pm} and 18\% with juvenile connective tissue disease–myositis overlap. anti-Ro52 autoantibodies were more frequent in patients with \gls{as} (64\%, p<0.001) and anti-MDA5 (31\%, p<0.05) autoantibodies. After controlling for the presence of myositis-specific autoantibodies, anti-Ro52 autoantibodies were associated with the presence of ILD (36\% vs 4\%, p<0.001). Disease course was more frequently chronic, remission was less common, and an increased number of medications was received in anti-Ro52 positive patients.

In conclussion, anti-Ro52 autoantibodies are present in 14\% of patients with juvenile myositis and are strongly associated with anti-MDA5 and \gls{as} autoantibodies. In all patients with juvenile myositis, those with anti-Ro52 autoantibodies were more likely to have \gls{ild}. Furthermore, patients with anti-Ro52 autoantibodies have more severe disease and a poorer prognosis.

\includepdf[pages=-]{pdf/ro52.pdf}

\section{More prominent muscle involvement in those dermatomyositis patients with anti-Mi2 autoantibodies. Neurology 2019.}
\label{sec:mi2_clinical}

In this longitudinal cohort study we aimed to define the clinical phenotype of dermatomyositis (DM) with anti-Mi2 autoantibodies. We analyzed the prevalence and severity of clinical features at disease onset and during follow-up in patients with anti-Mi2-positive \gls{dm} compared to patients with anti-Mi2-negative \gls{dm}, \gls{as}, and \gls{imnm}. We also assessed the longitudinal titers of anti-Mi2 autoantibody.

A total of 58 patients with anti-Mi2-positive \gls{dm}, 143 patients with anti-Mi2-negative \gls{dm}, 162 patients with \gls{as}, and 170 patients with \gls{imnm} were included. Among patients with anti-Mi2-positive \gls{dm}, muscle weakness was present in 60\% at disease onset and occurred in 98\% during longitudinal follow-up; fewer patients with anti-Mi2-negative \gls{dm} developed weakness (85\%; p = 0.008). Patients with anti-Mi2-positive \gls{dm} were weaker and had higher \gls{ck} levels than patients with anti-Mi2-negative \gls{dm} or patients with \gls{as}. Muscle biopsies from patients with anti-Mi2-positive \gls{dm} had prominent necrosis. Anti-Mi2 autoantibody levels correlated with \gls{ck} levels and strength (p < 0.001). With treatment, most patients with anti-Mi2-positive \gls{dm} had improved strength and \gls{ck} levels; among 10 with multiple serum samples collected over 4 or more years, anti-Mi2 autoantibody titers declined in all and normalized in 3, 2 of whom stopped immunosuppressant treatment and never relapsed. Patients with anti-Mi2-positive \gls{dm} had less calcinosis (9\% vs 28\%; p = 0.003), interstitial lung disease (5\% vs 16\%; p = 0.04), and fever (7\% vs 21\%; p = 0.02) than did patients with anti-Mi2- negative \gls{dm}.

In conclussion, patients with anti-Mi2-positive \gls{dm} have more severe muscle disease than patients with anti-Mi2-negative \gls{dm} or patients with \gls{as}. Anti-Mi2 autoantibody levels correlate with disease severity and may normalize in patients who enter remission.

\includepdf[pages=-]{pdf/mi2_clinical.pdf}

\section{Validation of anti-Mi2 autoantibody testing by line blot. Autoimmun Rev 2020.}
\label{sec:mi2_serologic}

Immunoprecipitation is the gold standard for detecting anti-Mi2 autoantibodies. The objective of this study was to assess the performance of a commercially available anti-Mi2 autoantibody line blot test.

To do this we included all the patients from the Johns Hopkins Myositis Center tested for anti-Mi2$\alpha$/$\beta$ autoantibodies by immunoprecipitation (which tests for both anti-Mi2$\alpha$ and anti-Mi2$\beta$ autoantibodies but does not distinguish between the two) and line blot (which tests separately for anti-Mi2$\alpha$ and anti-Mi2$\beta$ autoantibodies).  Patients who were anti-Mi2$\alpha$ and/or Mi2$\beta$ -positive by IP and/or line blot had sera tested for anti-Mi2$\beta$  autoantibodies by ELISA.     

Among 666 patients, 35 (5\%) were anti-Mi2$\alpha$/$\beta$ positive by IP. From among these, by line blot, 71\% (n=25) were positive for both anti-Mi2$\alpha$ and anti-Mi2$\beta$, 23\% (n=8) were exclusively positive for anti-Mi2$\alpha$, and 6\% (n=2) were exclusively positive for anti-Mi2$\beta$.  False positive line blot results occurred in 5\% (n=34) of patients, including 3\% (n=1) for both anti-Mi2$\alpha$ and anti-Mi2$\beta$, 15\% (n=5) for anti-Mi2$\alpha$ alone, and 82\% (n=28) for anti-Mi2$\beta$ alone.  The signal intensity of anti-Mi2$\alpha$ and anti-Mi2$\beta$ line blot tests correlated well with each other (r=0.77, p<0.001) and with anti-Mi2$\beta$ ELISA titers (both r>0.67, both p<0.001).  

In conclussion, a dual positive anti-Mi2$\alpha$ and anti-Mi2$\beta$ line blot test reliably identifies anti-Mi2-positive patients. Exclusive anti-Mi2$\beta$ line blot positive testing is most consistent with a false positive result. Those who are only positive for anti-Mi2$\alpha$ by line blot require validation by another technique. The line blot signal intensity can be used to estimate anti-Mi2$\beta$ autoantibody titers.


\includepdf[pages=-]{pdf/mi2_serologic.pdf}

\section{In adults with myositis-specific autoantibodies, autoantibodies outperform the 2017 EULAR/ACR classification criteria to define phenotypes. Submitted to Ann Rheum Dis 2020.}
\label{sec:atb_class}

The objective of this study was to evaluate the sensitivity of the 2017 EULAR/ACR criteria to classify inflammatory myopathy (IM) patients with \gls{msa} and to compare the performance of autoantibodies with the EULAR/ACR classification to predict clinical phenotype of \gls{msa}-positive patients.

For this we included 524 \gls{msa}-positive IM patients from the Johns Hopkins Myositis Center. Each patient was classified using the EULAR/ACR classification criteria. The evolution of muscle strength and creatine kinase levels was studied using \gls{lowess} and patient phenotypes were summarized using factor analysis of mixed data. We compared the ability of MSAs and the EULAR/ACR subgroups to predict the phenotype of patients by applying the \gls{aic} and the \gls{bic} to the linear regression models.

Nine percent of \gls{msa}-positive patients did not meet EULAR/ACR criteria to be classified as inflammatory myopathy. Anti-HMGCR (20\%), anti-SRP (9\%), anti-MDA5 (11\%), anti-PL12 (10\%) and anti-PL7 (50\%) patients had the highest failure rates. Around 10\% of anti-SRP and anti-HMGCR patients were misclassified as \gls{ibm} using EULAR/ACR criteria. \gls{lowess} showed a characteristic evolution of the muscle involvement in each \gls{msa} group and factor analysis of mixed data demonstrated that patients within each \gls{msa} group had similar phenotypes. Application of both the \gls{aic} and \gls{bic} to the linear regression models revealed that \gls{msa} better predict myositis phenotypes than the subgroups defined by the EULAR/ACR criteria.

In conclusions, \gls{msa} outperform the 2017 EULAR/ACR classification to predict the clinical phenotypes of myositis patients. Thus, we propose using \gls{msa} to build phenotypically homogeneous groups in myositis research.

\includepdf[pages=-]{pdf/atb_class.pdf}

\section{Myositis autoantigen expression correlates with muscle regeneration but not autoantibody specificity. A\&R 2019.}
\label{sec:autoantigens}

Although more than a dozen \gls{msa} have been identified, most patients with myositis are positive for a single \gls{msa}. The specific overexpression of a given myositis autoantigen in myositis muscle has been proposed as initiating and/or propagating autoimmunity against that particular autoantigen. The present study was undertaken to test this hypothesis.

In order to quantify autoantigen RNA expression, RNA sequencing was performed on muscle biopsy samples from control subjects, \gls{msa}-positive patients with myositis, regenerating mouse muscles, and cultured human muscle cells.

Muscle biopsy samples were available from 20 control subjects and 106 patients with autoantibodies recognizing HMGCR (n = 40), SRP (n = 9), Jo-1 (n = 18), NXP2 (n = 12), Mi-2 (n = 11), TIF1$\gamma$ (n = 11), or MDA5 (n = 5). The increased expression of a given autoantigen in myositis muscle was not associated with autoantibodies recognizing that autoantigen (all q > 0.05). In biopsy specimens from both myositis muscle and regenerating mouse muscles, autoantigen expression correlated directly with the expression of muscle regeneration markers and correlated inversely with the expression of genes encoding mature muscle proteins. Myositis autoantigens were also expressed at high levels in cultured human muscle cells.

In conclussion, most myositis autoantigens are highly expressed during muscle regeneration, which may relate to the propagation of autoimmunity. However, factors other than overexpression of specific autoantigens are likely to govern the development of unique autoantibodies in individual patients with myositis.

\includepdf[pages=-]{pdf/autoantigens.pdf}

\section{Identification of distinctive interferon gene signatures in different types of myositis. Neurology 2019.}
\label{sec:ifn}

Activation of the type 1 \gls{ifn} pathway is a prominent feature of \gls{dm} muscle and may play a role in the pathogenesis of this disease. However, the relevance of the \gls{ifn}1 pathway in patients with other types of myositis such as the \gls{as}, \gls{imnm}, and \gls{ibm} is largely unknown. Moreover, the activation of the type 2 \gls{ifn} pathway has not been comprehensively explored in myositis. In this cross-sectional study, our objective was to determine whether \gls{ifn}1 and \gls{ifn}2 pathways are differentially activated in different types of myositis by performing RNA sequencing on muscle biopsy samples from 119 patients with \gls{dm}, \gls{imnm}, \gls{as}, or \gls{ibm} and on 20 normal muscle biopsies.

The expression of \gls{ifn}1-inducible genes was high in \gls{dm}, moderate in \gls{as}, and low in \gls{imnm} and \gls{ibm}. In contrast, the expression of \gls{ifn}2-inducible genes was high in \gls{dm}, \gls{ibm}, and \gls{as} but low in \gls{imnm}. The expression of \gls{ifn}-inducible genes correlated with the expression of genes associated with inflammation and muscle regeneration. Of note, ISG15 expression levels alone performed as well as composite scores relying on multiple genes to monitor activation of the \gls{ifn}1 pathway in myositis muscle biopsies.

In conclussion, \gls{ifn}1 and \gls{ifn}2 pathways are differentially activated in different forms of myositis. This observation may have therapeutic implications because immunosuppressive medications may preferentially target each of these pathways.

\includepdf[pages=-]{pdf/ifn.pdf}

\section{Machine learning algorithms reveal unique gene expression profiles in muscle biopsies from patients with different types of myositis. ARD 2020.}
\label{sec:rnaseq_ml}

The purpose of this study was to define unique gene expression profiles in muscle biopsies from patients with \gls{msa}-positive \gls{dm}, \gls{as} and \gls{imnm} as well as \gls{ibm}.

RNA-seq was performed on muscle biopsies from 119 myositis patients with \gls{ibm} or defined \gls{msa}s and 20 controls. Machine learning algorithms were trained on transcriptomic data and recursive feature elimination was used to determine which genes were most useful for classifying muscle biopsies into each type and MSA-defined subtype of myositis.

The support vector machine learning algorithm classified the muscle biopsies with >90\% accuracy. Recursive feature elimination identified genes that are most useful to the machine learning algorithm and that are only overexpressed in one type of myositis.
For example, CAMK1G (calcium/calmodulin-dependent protein kinase IG), EGR4 (early growth response protein 4) and CXCL8 (interleukin 8) are highly expressed in \gls{as} but not in \gls{dm} or other types of myositis. Using the same computational approach, we also identified genes that are uniquely overexpressed in different \gls{msa}-defined subtypes. These included apolipoprotein A4 (APOA4), which is only expressed in HMGCR myopathy, and MADCAM1 (mucosal vascular addressin cell adhesion molecule 1), which is only expressed in anti-Mi2-positive \gls{dm}.

In conclusion, unique gene expression profiles in muscle biopsies from patients with \gls{msa}-defined subtypes of myositis and \gls{ibm} suggest that different pathological mechanisms underly muscle damage in each of these diseases.

\includepdf[pages=-]{pdf/rnaseq_ml.pdf}