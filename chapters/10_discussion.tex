\chapter{Discussion}

\section{Controversies on the classification}
However, nowadays it is increasingly more common to design clinical trials in myositis focused in specific types of myostis subgroups, since it is obvious that the different types of patients do not respond equally (or at all) to the different immunosuppressant treatments. The logical conclusion of this is that criteria should prioritize defining homogeneous subsets of patients with myositis. Also, it is increasingly complicated to justify methodologically how to define diseases that do not share any clinical feature with a single set of criteria. For example, patients positive for the anti-MDA5 autoantibodies usually present with clinically amyopathic dermatomyositis and rapidly progressive interstitial lung disease whereas patients with \gls{ibm} present with a particular pattern of muscle involvement with no skin or lung manifestations. These two subsets of patients have clearly very different syndromes and it does not seem very rational to try to develop criteria for both of them simultaneously just because other patients with \gls{dm} have both muscle and skin involvement.

\section{U1-RNP myositis}

In this study, we have defined the distinctive clinical phenotype of myositis patients with anti-U1-RNP autoantibodies.  These patients are younger and more likely to be black then those with DM, IMNM, or AS.  This is consistent with a prior report that a high proportion of anti-U1-RNP-positive patients are black 126.  Like the other myositis groups, those with anti-U1-RNP autoantibodies have proximal pattern of muscle weakness.  However, the neck muscles are spared only in those with anti-U1-RNP autoantibodies.  Anti-U1-RNP-positive patients are also notable for their prominent extramuscular manifestations.  These include Raynaud phenomenon (80\%), arthralgia/arthritis (60\%), DM skin features (60\%), necrotizing muscle biopsies (57\%), mechanic’s hands (50\%), and dysphagia (50\%).  Other common clinical manifestations in these patients include ILD (45\%), pericarditis (40\%), subcutaneous edema (35\%), fever (35\%), glomerulonephritis (25\%), pulmonary hypertension (25\%), sclerodactyly (25\%), and calcinosis (25\%).  

The unique clinical phenotype of anti-U1-RNP-positive patients can be further appreciated by comparing them to each of the three other myositis groups separately.  Compared to DM patients, those with anti-U1-RNP autoantibodies are more likely to have sclerodactyly, Raynaud phenomenon, mechanic’s hands, ILD, arthritis, pericarditis, and glomerulonephritis; as expected, they are less likely to have heliotrope or Gottron sign.  Compared with AS patients, anti-U1-RNP-positive patients are more likely to have Raynaud phenomenon, dysphagia, pericarditis, and glomerulonephritis; they are less likely to have ILD and sicca syndrome.  Finally, anti-U1-RNP-positive patients are more likely to have all of the studied extramuscular manifestations of disease compared to those with IMNM.  In contrast, significantly more IMNM patients have weakness and weakness is more severe in IMNM patients.  

Given that IMNM seems to have the least in common with anti-U1-RNP-positive patients, it may be surprising to find that muscle biopsies from both groups can be strikingly similar with prominent myofiber necrosis and scant lymphocytic infiltration.  Of note, others have also reported myofiber necrosis and regeneration in muscle biopsies from anti-U1-RNP-positive patients 127,128.  Since the prognosis of patients with anti-U1-RNP autoantibodies is very different than those with anti-HMGCR myopathy or anti-SRP myopathy, testing for each of these autoantibodies is indicated in patients presenting with a necrotizing muscle biopsy. 

Pericarditis with or without glomerulonephritis occurred in 40\% of myositis patients with anti-U1-RNP autoantibodies; these complications were exceedingly rare in the other myositis groups.  Of note, all 8 anti-U1-RNP-positive myositis patients with pericarditis/glomerulonephritis had co-existing anti-Ro52 autoantibodies.  Similarly, PH was detected only in RNP-positive patients who were also positive for anti-Ro52 autoantibodies.  Taken together, 60\% of anti-U1-RNP-positive patients with co-existing anti-Ro52 antibodies developed pulmonary hypertension, pericarditis and/or glomerulonephritis, while no patient without anti-Ro52 developed any of these manifestations (p=0.04).  Although it requires confirmation in other cohorts, based on these observations, clinicians could consider testing anti-U1-RNP-positive myositis patients for anti-Ro52 to identify those patients most at risk for developing these serious extramuscular manifestations of disease.

This study has several limitations. First, most of the conclusions are based on signs and symptoms that were recorded prospectively from the beginning of the study in 2002. Consequently, we could not include activity and damage tools that were not available when the study started. Second, data used in this study is based on patients presenting to a multidisciplinary Myositis Center and may be biased towards including patients with active muscle and lung disease. Third, due to the relatively small sample size of our anti-U1-RNP-positive population, our study may have been underpowered to detect differences in some key features like the association between ILD and anti-Ro52 autoantibodies.  Future studies including larger numbers of anti-U1-RNP-positive patients will be of value. 

These limitations notwithstanding, we have shown that patients with anti-U1-RNP autoantibodies appear to have a unique syndrome different from patients with DM, AS, or IMNM.  This syndrome is characterized by proximal muscle weakness, necrotizing muscle biopsies, and frequent extramuscular manifestations.  Glomerulonephritis, pericarditis, and pulmonary hypertension are relatively common in anti-U1-RNP-positive patients with co-existing anti-Ro52 autoantibodies but rare in the other myositis groups. We propose that testing for anti-Ro52 autoantibodies may be useful to determine which anti-U1-RNP-positive patients are at most risk for these complications.

\section{MTX vs. AZA}

This study demonstrated that MTX and AZA are comparable in terms of rate of muscle strength recovery, rate of corticosteroid tapering, and rate of CK decrease with similar rates of adverse events. We did identify two episodes of MTX pneumonitis which reversed with discontinuation of therapy.

MTX has been reported to cause pneumonitis in 4-8\% of the patients exposed to this drug129 and this may dissuade clinicians from prescribing MTX in patients with ASyS autoantibodies or preexisting ILD.130 Our study confirms previous data regarding the low prevalence of MTX pneumonitis (4\%). 

Some authors have suggested an increased efficacy of MTX over AZA in selected groups of patients.131 Our study found that MTX was comparable to AZA in terms of efficacy in patients with the ASyS.

The data that we report is based on a cohort study followed longitudinally in the context of routine clinical care and not a clinical trial. The assignment of therapy to the individual patient was based on physician preference, thus, it is possible that some of the analyses were subject to unaccounted bias. Patients underwent PFTs and CT imaging as part of clinical care, therefore, we cannot comment on the appearance of some features such as ILD except as detected based on clinical symptoms and findings Likewise, adverse events were both patient-reported and surveyed by the treating clinicians but not necessarily in a routine manner for all patients. Moreover, the small number of patients in each group precludes a cautious interpretation of our results.

In conclusion, in our real-world clinical experience, we found that compared with AZA, MTX had a similar prevalence of adverse effects and efficacy. MTX pneumonitis occurred in 4\% of patients started on this medication, but was entirely reversible with stopping therapy, thus, attention to this potential adverse event is important with rapid discontinuation of therapy if symptoms occur.

\section{Anti-Ro52}
Here, we utilized a large cohort of juvenile myositis patients to study the prevalence and clinical significance of anti-Ro52 autoantibodies in children with IIM. We found anti-Ro52 autoantibodies to be strongly associated with ILD and other pulmonary manifestations in juvenile myositis patients. We also found that children with anti-Ro52 autoantibodies have more severe disease, underwent more intense treatment regimens, and had lower rates of disease remission than those without anti-Ro52 autoantibodies.  In children with myositis, anti-Ro52 autoantibodies were associated with anti-aminoacyl tRNA synthetase autoantibodies, as previously described in adults.49 We also found that anti-Ro52 autoantibodies were associated with anti-MDA5 autoantibodies in pediatric myositis patients, which has not been reported previously.  

Importantly, our analyses indicate that the presence of anti-Ro52 autoantibodies is strongly associated with ILD, even after adjusting for the presence of MSAs such as anti-MDA5 and anti-aminoacyl tRNA synthetase autoantibodies. Indeed, the association of Ro52 reactivity with ILD is not limited to the anti-MDA5 and anti-aminoacyl tRNA synthetase autoantibody subgroups, but extends to other MSA subgroups that are not classically associated with ILD, such as children with anti-p155/140 (TIF-1) autoantibodies.  Interestingly, none of the 5 Ro52-positive MSA-negative patients had ILD; however, the small number of patients limits our ability to draw definite conclusions about whether anti-Ro52 autoantibodies are associated with ILD in this subgroup.  Current practice encourages screening juvenile myositis patients for MSAs such as anti-MDA5 and anti-aminoacyl tRNA synthetase autoantibodies, as these autoantibodies confer risk for developing ILD and their presence is a determinant of clinical management and patient prognosis. In light of the current findings demonstrating that anti-Ro52 autoantibodies are an independent predictor of ILD, screening juvenile myositis patients for these autoantibodies may also be prudent.

In adult patients with IIM, anti-Ro52 autoantibodies have been associated with poorer response to immunosuppressive drugs and decreased survival.132,133 Similarly, in our juvenile cohort, anti-Ro52 autoantibodies are associated with more severe disease and poorer outcomes. As the severity of other clinical manifestations, including muscle, joint, skin, gastrointestinal, and systemic features were not associated with Ro52 reactivity, it seems likely that disease severity seen in the anti-Ro52 positive patients is a consequence of pulmonary disease. Additional studies are required to clarify this point.  Nonetheless, our findings highlight the potential utility of anti-Ro52 autoantibodies as a predictor of disease severity and poor prognosis in juvenile myositis, which underscores the potential utility of screening juvenile IIM patients for anti-Ro52 autoantibodies. 

Of particular significance is the novel association of anti-Ro52 autoantibodies and anti-MDA5 autoantibodies in our JIIM cohort. In adult IIM patients, anti-Ro52 autoantibodies often co-occur with anti-Jo1 autoantibodies, and in adult anti-Jo1 positive patients, Ro52 reactivity is associated with more severe ILD. Until now neither the association of anti-Ro52 autoantibodies with anti-MDA5 autoantibodies, nor the association of anti-Ro52 autoantibodies with ILD across other MSA groups, has been observed in adult patients with myositis. However, a small case series described co-existing anti-Ro52 autoantibodies in 6 of 13 anti-MDA5 positive patients, 5 of whom had rapidly progressive ILD [Huang 2018].  Interestingly, only 1 of 33 patients in our JIIM cohort with ILD had rapidly progressive ILD and this patient was positive for both anti-MDA5 and anti-Ro52 autoantibodies. 

Although we have now established an association between anti-aminoacyl tRNA synthetase and anti-Ro52 autoantibodies not only in adults, but also in children, it remains unclear why these autoantibodies co-occur. It has been proposed that local autoantibody production induced by type I IFN134 could be a driving force behind the production of both anti-Jo1 and anti-Ro52 autoantibodies, given the increase in B-cell activating factor (BAFF) receptors in the sera of IIM patients with these autoantibodies.135 In the current study of juvenile IIM, we now also demonstrate an association between anti-MDA5 and anti-Ro52 autoantibodies. Interestingly, both MDA5 and Ro52 are cytosolic, interferon (IFN)-induced proteins; perhaps concurrent over-expression of these proteins in juvenile IIM patients leads to the development of autoimmunity against both. However, we do not have adequate type I IFN measurements to further examine this hypothesis.

This current study has several limitations.  First, this cohort of patients with juvenile myositis had some data collected retrospectively, resulting in some missing data, and was collected over more than 20 years, with potential chronology bias. However, we adjusted the variables of this study for the year of diagnosis and tested the distribution of missing values across groups and did not find evidence of a significant bias. Second, although imaging studies were available to confirm the diagnosis of ILD in more than 90\% of patients who had ILD, pulmonary function testing data were not available for many of the patients, as a number of the children were of young age when such testing is unreliable in children. Thus, we were not able to study whether ILD patients with anti-Ro52 autoantibodies had more severe pulmonary dysfunction than those without these autoantibodies. Also, we cannot confirm the absence of ILD as many of the children without clinical suspicion of ILD did not have imaging and/or pulmonary function testing. This however, is a limitation of standard clinical care in pediatric patients who have challenges to undergo such testing.

Overall, this study shows that anti-Ro52 autoantibodies are present in 14\% of patients with juvenile myositis and are strongly associated with ILD, more severe illness, and poorer outcomes, even when correcting for the co-existence of MSAs. In juvenile myositis patients, anti-Ro52 autoantibodies are associated not only with the presence of anti-synthetase autoantibodies, as previously reported in adult myositis patients, but also with anti-MDA5 autoantibodies, and the co-existence of these MSAs increases the likelihood of ILD and poor outcome. The current standard of care in patients with juvenile myositis who have reactivity to MSAs associated with pulmonary manifestations (such as anti-MDA5 and anti-aminoacyl tRNA synthetase autoantibodies) is to have a high index of suspicion for the development of ILD and modify management accordingly. Our data suggest that testing for anti-Ro52 autoantibodies may also have a role in disease monitoring, management, and patient prognosis in juvenile myositis patients. Future studies will be required to determine whether anti-Ro52 autoantibodies are not only useful biomarkers, but whether they also play a pathological role in the development of ILD and other disease manifestations in myositis patients.

\section{Anti-Mi2 serological}
New immunologic techniques like the line blot have made testing for autoantibodies both quick and easy. However, for rare autoimmune diseases like myositis, companies may have insufficient numbers of positive and negative control samples to properly validate their assays. Although a recent study assessed the utility of line blot compared to IP to detect myositis-specific autoantibodies136, the small number of samples from patients with anti-Mi2 autoantibodies included in the study precluded drawing conclusions about the utility of the anti-Mi2$\alpha$ and anti-Mi2$\beta$ line blot test.136 

In this study, we have defined the diagnostic utility of the anti-Mi2$\alpha$ and anti-Mi2$\beta$ line blot tests for anti-Mi2 autoantibodies.  Our results indicate that the line blot test has a low positive predictive value if testing positive for only one of anti-Mi2$\alpha$ or anti-Mi2$\beta$ is considered positive. Indeed, our results show that only those subjects who are positive both for anti-Mi2$\alpha$ and anti-Mi2$\beta$ autoantibodies by line blot are very likely to be anti-Mi2 positive by IP. Those who are positive for just for anti-Mi2$\beta$ are usually false positive and those who are only positive for anti-Mi2$\alpha$ have only about a fifty percent chance of being true anti-Mi2 positives; further testing of the latter patients is required. These results, testing a large number of samples from patients with a rare disease, underscore the necessity for standardization of myositis autoantibody testing techniques for both clinical and research purposes. 

We also demonstrate that that anti-Mi2 line blot signal intensities can be used to accurately estimate anti-Mi2 autoantibody titers. This may be clinically relevant, since anti-Mi2 autoantibody titers may be useful markers of disease activity. The fact that a readily available technique like line blot could be used to measure autoantibody titers has the potential of simplifying longitudinal autoantibody studies.

In summary, only those patients positive for both anti-Mi2$\alpha$ and anti-Mi2$\beta$ by line blot can reliably be considered anti-Mi2 without further validation. Moreover, line blot signal intensity has quantitative value to estimate anti-Mi2 autoantibody titers.

\section{Myositis autoantigens}

As several myositis autoantigens (i.e., Mi2, TIF1$\gamma$ and Jo1) were previously shown to be expressed at high levels in regenerating muscle cells, it has been proposed that the overexpression of specific autoantigens in myositis muscle might drive the autoantigen-specific immune response.137 In this study, we used RNAseq to systematically investigate autoantigen expression levels in muscle biopsies from myositis patients with each major MSA.  We found that RNA levels of each myositis autoantigen are positively correlated with markers of muscle regeneration but that the levels of a given autoantigen are not associated with the presence of the corresponding autoantibody.  Therefore, restricted autoantigen overexpression alone does not account for why myositis patients typically produce only a single MSA. Rather, it is likely that factors such as aberrant post-translational processing,138 mislocalization of autoantigen, immunogenetic susceptibility,36 and/or exposure to molecularly similar antigens (e.g. tumor antigens38 determine which autoantigens will be targeted by the immune system in myositis patients. 

We also showed that all myositis autoantigens are expressed at high levels not just in regenerating myositis muscle, but also in regenerating mouse muscles and in cultured human myoblasts.  This indicates that elevated myositis autoantigen expression is a normal part of muscle regeneration/differentiation.  Nonetheless, disease-related factors may also contribute to the myositis autoantigen overexpression.  For example, IFIH1 is expressed at low levels (<2) during all phases of cultured muscle cell differentiation compared to the expression levels of other myositis autoantigens (4-6).  However, IFIH1 is expressed at especially high levels in muscle biopsies from patients with DM autoantibodies.  Since interferon (IFN) levels are high in DM patients 39 and IFIH1 is an IFN-inducible gene, we hypothesize that muscle regeneration and IFN both contribute to the high levels of IFIH1 in DM muscle biopsies.   

The primary limitation of this study is that we relied on RNA quantitation to assess gene expression levels.  However, the utilization of high-throughput next-generation sequencing is also what allowed us to analyze the expression of many genes in each of many samples.  For example, figure 2 summarizes the expression levels of 20 genes in over 106 myositis muscle biopsies and 26 mouse muscle specimens.  Such an analysis would be impractical using immunoblotting techniques to quantify protein expression levels.  Furthermore, we and others have previously shown that Mi2, TIF1$\gamma$, Jo1, HMGCR and SRP proteins are upregulated in regenerating cells of myositis muscle biopsies,46,57,137,139,140 validating a correlation between RNA and protein levels for these autoantigens.

In summary, by utilizing RNAseq to quantitate autoantigen expression in a large number of myositis muscle biopsies from patients with defined autoantibodies, we have demonstrated that autoantigen expression is highly correlated with muscle regeneration but that expression of a given autoantigen is not associated with the presence of the corresponding autoantibody. Future studies will be required to determine why only one autoantigen is typically targeted by the immune system in a given myositis patient.

\section{Interferon myositis}
In this study, using RNAseq data from a large number of myositis and comparator muscle biopsies, we have established that the IFN1 pathway is activated, not only in DM patients as previously described,39,141-143 but also in patients with AS, IMNM, and IBM. Quantitatively, the IFN1 pathway was most up-regulated in DM, with intermediate activation of the pathway in AS and lower levels of activation in IBM and IMNM. We also used RNAseq data to study activation of the IFN2 pathway, demonstrating robust activation in AS, IBM, and DM, but not in IMNM. We were also able to show that activation of the IFN pathway was associated with increased expression of inflammatory cell and muscle regeneration genes. Finally, we established that ISG15 gene expression can be used as a surrogate marker of IFN1 pathway activation in myositis since it performs as well as a more complex composite score.

Interestingly, different collections of IFN-inducible genes were most prominently upregulated in the different groups. For example, the IFN1 genes ISG15, IFI6, and MX1 were the most upregulated IFN-inducible genes in DM. In contrast, IFI30, NCAM1, and SOCS3 were the most upregulated IFN-inducible genes in IMNM patients. Of note, the IFN2 genes PSMB8, GBP2, and GBP1 were the most upregulated IFN-inducible genes in both AS and IBM patients, underscoring the prominence of the IFN2 pathway in these two diseases. Taken together, this suggests that the degree of activation of the IFN1 inflammatory pathways differs between DM, IBM, AS and IMNM, and most but not all types of myositis involve the IFN2 inflammatory pathways.

It is well-established that DM patients with different myositis autoantibodies have unique clinical manifestations. In fact, there are differences in muscle biopsy features between DM patients with different autoantibodies.22 For example, half of the muscle biopsies from anti-Mi2-positive DM patients include examples of lymphocytes surrounding and invading healthy muscle fibers; this histopathologic feature was never seen in DM patients with anti-NXP2 autoantibodies. Despite these histopathologic differences, the IFN gene signature was remarkably similar between DM patients with different myositis autoantibodies. Indeed, ISG15 and IFI6 were the top two IFN-inducible genes in each of the serologically defined DM subgroups and MX1 and MX2 present among the top ten IFN-inducible genes in each DM subgroup. These findings suggest that, at least with regard to activation of IFN pathways in the muscle, the different autoantibody subgroups of DM are more alike than different. Similarly, in IMNM patients with either anti-SRP or -HMGCR autoantibodies, IFI30, NCAM1, VCAM1, ICAM1, SOC3, GBP2, and MT2A were among the top ten IFN-inducible genes. We did not have a sufficient number of biopsies from patients with anti-PL7, anti-PL12, or other non-Jo1 antisynthetase autoantibodies to determine whether these serologic subgroups of the anti-synthetase syndrome share a similar IFN gene signature pattern.

Some investigators have shown that immunostaining muscle biopsies for specific IFN-inducible proteins can be used to distinguish between different types of myositis. For example, DM but not AS muscle biopsies stain positive for MxA (MX1)78 or RIG-I (DDX58),21 both IFN1 inducible genes. Our RNAseq data, which shows higher expression levels of these genes in DM than in AS (MX1 fold-change 4.7 and RIG-1 fold-change 3.3, both q-values<5•10-9), is consistent with this observation. Also, ISG15 overexpression was reported to be useful to diagnose patients with DM and perifascicular atrophy.144 Accordingly, we found a marked preferential overexpression of ISG15 in DM patients (ISG15 fold-change compared to comparator biopsies 101, q-value 1.1•10-91).

As it was mentioned earlier, in this study we determined that ISG15 expression levels alone can be used to reliably quantitate the activation of the IFN1 pathway in myositis muscle biopsies. In fact, measuring ISG15 levels was equivalent to a composite score derived from measuring expression levels of 13 different IFN genes, which is concordant with previous data showing the marked specificity of ISG15 muscle transcript measurements for DM with perifascicular atrophy.144 Also, we noted that either ISG15 expression levels or the previously proposed composite IFN1 scores were useful for assessing activation of the IFN2 pathway in DM, but not in IBM or AS. Rather, directly measuring the expression levels of IFN2-inducible genes such as PSMB8, GBP1 or GBP2 may be required. 

This study has several limitations. For example, some less common autoantibody groups (e.g., non-anti-Jo1 AS patients) could not be included due to an insufficient number of biopsies. In addition, we only had relevant CK and strength information for muscle biopsies obtained at Johns Hopkins, which may have limited our ability to show significant associations between IFN pathway activation and markers of clinical disease activity, such as strength and CK levels.

In conclusion, this study demonstrates that DM muscle biopsies are characterized by high levels of both IFN1- and IFN2-inducible genes. In contrast, biopsies from patients with AS and IBM reveal gene expression patterns consistent with prominent IFN2 activation. Finally, RNAseq analysis reveals IMNM biopsies show relatively low activation of the interferon pathway. These findings are consistent with recent case series suggesting the efficacy of JAK/STAT inhibitors in patients with DM.107,108,145-147 They also suggest that these agents may be effective in patients with AS or IBM. However, the relatively modest activation of IFN pathways in IMNM does not provide compelling evidence to support the use of JAK/STAT inhibitors in this patient population.

