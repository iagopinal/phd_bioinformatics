\chapter{Conclusions}

\begin{itemize}
	
	\item We developed a research framework that was able to generate an efficient research pipeline to study longitudinal cohorts of specific myositis subgroups. With this framework, we could prove that:
	
	\begin{itemize}
		\item Patients with anti-U1RNP myositis typically present with proximal weakness and necrotizing muscle biopsies. Arthritis, dermatitis, and ILD are the most common extramuscular clinical features. Pericarditis and glomerulonephritis are uniquely found in patients with anti-U1-RNP- positive myositis.
		
		\item Azathioprine and methotrexate have similar efficacy and adverse events in patients with \gls{as}. Pneumonitis is a rare but important event in patients receiving methotrexate.
		
		\item anti-Ro52 autoantibodies are present in 14\% of patients with juvenile myositis and are strongly associated with anti-MDA5 and antisynthetase autoantibodies. Anti-Ro52 autoantibodies in juvenile myositis are strongly associated with ILD. Furthermore, patients with anti-Ro52 autoantibodies have more severe disease and poorer prognosis.
		
		\item Patients with anti-Mi2-positive \gls{dm} have more severe muscle disease than patients with anti-Mi2-negative \gls{dm} or patients with \gls{as}. Anti-Mi2 autoantibody levels correlate with disease severity and may normalize in patients who enter remission.
		
		\item Only those subjects that are positive for both anti-Mi2$\alpha$ and anti-Mi2$\beta$ by line blot can be reliably considered anti-Mi2 without further validation.
		
		\item \gls{msa} outperform the 2017 EULAR/ACR criteria to predict the phenotype of patients with myositis.
		
		\item \gls{msa} can be used to build phenotypically homogeneous groups in myositis research.
	\end{itemize}

	\item We performed a systematic analysis of the transcriptome of muscle biopsies from patients with different types of myositis showing that:
	\begin{itemize}
		\item Most myositis autoantigens are highly expressed during muscle regeneration but are not associated with autoantibody specificity.
		
		\item \gls{ifn}1 is most upregulated in \gls{dm}, with intermediate activation of the pathway in \gls{as} and lower levels of activation in \gls{ibm} and \gls{imnm}. Alternatively, \gls{ifn}2 is robustly activated in \gls{dm}, \gls{as}, and \gls{ibm} but not in \gls{imnm}.
		
		\item Unique gene expression profiles in muscle biopsies from patients with \gls{msa}-defined subtypes of myositis and \gls{ibm} suggest that different pathological mechanisms underly muscle damage in each of these diseases.
	\end{itemize}
\end{itemize}
