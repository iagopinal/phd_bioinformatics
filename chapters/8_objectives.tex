\chapter{Motivation and Objectives}

\section{Motivation}
The inflammatory myopathies are a heterogeneous family of rare diseases affecting multiple organs and systems, including the muscle, skin, lung, and/or the joints. Accurately defining its pathogenesis and classifying them are key to understand and manage these diseases.

Studying myositis pathogenesis based on an incorrect classification of its different individual diseases may lead to incorrect conclusions. For this reason, we believed that it was necessary to approach defining the myositis classification and its pathogenesis in parallel.

We hypothesize that disease-specific autoantibodies in myositis are tightly associated with the cause of the disease. If this is the case, different autoantibody groups will show distinct activation of pathogenic pathways, clinical manifestations, prognosis, and response to treatment.

Moreover, other autoantibodies not associated to a specific clinical syndrome may act as disease modifiers and increase the risk of developing certain clinical manifestations.

Finally, even if no disease-specific autoantibodies have been identified in \gls{ibm}, its clinical and epidemiological features are characteristic enough that we can study its pathogenesis separately from other types of myositis.

\section{Objectives}
\begin{itemize}
	
	\item To develop a research framework to study longitudinal cohorts of specific myositis subgroups. With this framework we will:
	\begin{itemize}
		\item Determine if the autoantibodies are superior to current clinical classification systems to predict the phenotype of patients with myositis.
		\item Study the characteristic clinical features, prognosis, and response to therapy of patients with different myositis autoantibodies.
	\end{itemize}
	\item Define which are the most important pathogenic pathways and the specific expression profiles in the muscle tissue of patients with different types of myositis.
\end{itemize}

